\documentclass[12pt]{report}

\usepackage[italian]{babel}
\usepackage[latin1]{inputenc}
\usepackage{url}
\usepackage{biblatex}
\usepackage{amsmath}


\bibliography{bib}

\title {IaasFog}
\author{Stefano Cadario, Luca Cavazzana}
\data{\date}

\begin{document}
\maketitle

\tableofcontents

\chapter{Introduzione}

\chapter{Funzioni}

Viene qui presentata una panoramica delle funzioni realizzate. Per informazioni pi\`u dettagliate circa il loro utilizzo consultare l'\verb|help| delle funzioni~\cite{lucaskanade81}.

\section[C++]{C\verb_++_}

\paragraph*{\verb_FindFeatures:_}

\paragraph*{\verb_iaasJoiningLine:_} calcola in coordinate omogenee la retta passante per due punti.

\section{Matlab}

\paragraph*{\verb_iaas:_}

\paragraph*{\verb_fogLevel:_} date le coordinate del punto di fuga calcola il livello della nebbia come il livello di grigio medio nell'intorno se questi \`e omogeneo.

\paragraph*{\verb_MichelsonContrast:_} date le coordinate di una feature e la relative immagine restituisce il livello di contrasto nell'intorno come $$\frac{I_{max}-I_{min}}{I_{max}+I_{min}}$$
\paragraph*{\verb_rsmContrast:_} date le coordinate di una feature e la relativa immagine calcola il livello di contrasto nell'intorno come $$\sqrt{\frac{1}{MN}\sum_{i=1}^N\sum_{j=1}^M(I_{ij}-\bar{I})^2}$$
\paragraph*{\verb_WeberContrast:_} date le coordinate di una feature e la relative immagine calcola il livello di contrasto come $$\frac{I-I_b}{I_b}$$ dove $I_b$ \`e il livello della nebbia.

\paragraph*{\verb_timeImpact:_} date le coordinate del punto di fuga, di una feature in due differenti immagini e il relativo lasso tempo, stima il tempo mancante affinch\'e la feature attraversi il piano focale. %TODO:giusto il termine?

\chapter{Installazione}
\section{OpenCV}
OpenCV (Open Source Computer Vision) \`e una libreria di funzioni per la computer vision, inizialmente sviluppata da Intel e ora distribuita sotto licenza open source. \`E possibile scaricare l'ultima versione all'indirizzo \url{http://sourceforge.net/projects/opencvlibrary/} \footnote{guida d'installazione ufficiale all'indirizzo \url{http://opencv.willowgarage.com/wiki/InstallGuide}}.

\subsection{Windows}
Per compilare la libreria \`e necessario installare degli header offerti da MinGW (\url{http://http://www.mingw.org/}) e CMake (\url{http://www.cmake.org/cmake/resources/software.html}).\\
\noindent Una volta assicuratisi che il path dei binari di MinGW \`e fra le variabili di ambiente configurare OpenCV mediante l'interfaccia di CMake.

\subsection{Linux / MacOSX}

\noindent Per l'installazione \`e necessario disporre di \verb|cmake|, inoltre devono essere soddisfatte le seguenti dipendenze:
\begin{itemize}
\item \verb|ffmpeg|
\item \verb|libxine-ffmpeg|
\item \verb|libavcodec-dev|
\item \verb|pgk-config|
\item \verb|libgtk2.0-dev|
\end{itemize}

\noindent reperibili mediante \verb|apt-get| o package manager.\\

\noindent Una volta scaricata e scompattata l'ultima versione di OpenCV (2.2.0 alla stesura del presente documento), creare una cartella dove verranno salvate le librerie configurate mediante \verb|cmake| e aprirla con una finestra di terminale. Nel nostro esempio la creeremo nella stessa cartella scompattata

\begin{verbatim}
	$ cd OpenCV2.2.0/
	$ mkdir release; cd release
\end{verbatim}

\noindent Lanciare quindi il comando

\begin{verbatim}
	$ cmake -D CMAKE_BUILD_TYPE=RELEASE -D CMAKE_INSTALL_PREFIX=/usr/local
	-D BUILD_PYTHON_SUPPORT=ON -D BUILD_EXAMPLES=ON ..
\end{verbatim}

\noindent per configurare la libreria. Il valore della flag \verb|CMAKE_INSTALL_PREFIX| sar\`a l'indirizzo in cui si vorr\`a poi installare OpenCV. Se i sorgenti non dovessero trovarsi nella directory superiore, sostituire i due punti con il path corretto.\\

\noindent A questo punto non rimane che compilare ed installare le librerie mediante i comandi

\begin{verbatim}
	$ make
	$ make install
\end{verbatim}

\noindent ed esportare il path nelle variabili di ambiente con il comando

\begin{verbatim}
	$ export LD_LIBRARY_PATH=/usr/local/lib/:$LD_LIBRARY_PATH
	$ sudo ldconfig
\end{verbatim}

\noindent Se invece si preferisce non installare le librerie, esportare direttamente il path 

\begin{verbatim}
	$ export LD_LIBRARY_PATH=<opencv_dir>:$LD_LIBRARY_PATH
	$ sudo ldconfig
\end{verbatim}

\noindent dove \verb|<opencv_dir>| \`e la cartella dove abbiamo compilato il le librerie, nel nostro caso \verb|~/OpenCV-2.2.0/release|.

\section{Installazione progetto}
\noindent Il progetto \`e disponibile via SVN alla pagina \url{http://code.google.com/p/iaasfog}.\\
Importare i sorgenti nella cartella \verb|c++| in Eclipse, importare gli header delle funzioni di OpenCV mediante Project $\rightarrow$ Properties  $\rightarrow$ C\slash C\verb|++| Editor $\rightarrow$ Settings $\rightarrow$ GCC C\verb|++| Compiler $\rightarrow$ Directories aggiungendo il percorso \verb|/usr/local/include/| e sotto GCC C\slash C\verb|++| Linker $\rightarrow$ Libraries le librerie \verb|opencv_core|, \verb|opencv_video|, \verb|opencv_highgui| e \verb|opencv_imgproc| in \verb|/usr/local/lib| (o qualunque path sia stato utilizzato per l'installazione).\\
Compilare.\\

\noindent In Matlab aggiungere il path degli m-file ed assicurarsi che il percorso specificato dalla varibile \verb|exec_path| in \verb|iaas.m| corrisponda alla cartella dell'eseguibile del codice C\verb|++|.\\
Per eseguire il programma lanciare nella console di Matlab il comando \verb|iaas|.

\printbibliography

\end{document}
